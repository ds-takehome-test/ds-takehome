% Options for packages loaded elsewhere
\PassOptionsToPackage{unicode}{hyperref}
\PassOptionsToPackage{hyphens}{url}
%
\documentclass[
]{article}
\usepackage{amsmath,amssymb}
\usepackage{lmodern}
\usepackage{iftex}
\ifPDFTeX
  \usepackage[T1]{fontenc}
  \usepackage[utf8]{inputenc}
  \usepackage{textcomp} % provide euro and other symbols
\else % if luatex or xetex
  \usepackage{unicode-math}
  \defaultfontfeatures{Scale=MatchLowercase}
  \defaultfontfeatures[\rmfamily]{Ligatures=TeX,Scale=1}
\fi
% Use upquote if available, for straight quotes in verbatim environments
\IfFileExists{upquote.sty}{\usepackage{upquote}}{}
\IfFileExists{microtype.sty}{% use microtype if available
  \usepackage[]{microtype}
  \UseMicrotypeSet[protrusion]{basicmath} % disable protrusion for tt fonts
}{}
\makeatletter
\@ifundefined{KOMAClassName}{% if non-KOMA class
  \IfFileExists{parskip.sty}{%
    \usepackage{parskip}
  }{% else
    \setlength{\parindent}{0pt}
    \setlength{\parskip}{6pt plus 2pt minus 1pt}}
}{% if KOMA class
  \KOMAoptions{parskip=half}}
\makeatother
\usepackage{xcolor}
\IfFileExists{xurl.sty}{\usepackage{xurl}}{} % add URL line breaks if available
\IfFileExists{bookmark.sty}{\usepackage{bookmark}}{\usepackage{hyperref}}
\hypersetup{
  pdftitle={Pluralsight Model Building Take-Home},
  pdfauthor={William Rinauto},
  hidelinks,
  pdfcreator={LaTeX via pandoc}}
\urlstyle{same} % disable monospaced font for URLs
\usepackage[margin=1in]{geometry}
\usepackage{color}
\usepackage{fancyvrb}
\newcommand{\VerbBar}{|}
\newcommand{\VERB}{\Verb[commandchars=\\\{\}]}
\DefineVerbatimEnvironment{Highlighting}{Verbatim}{commandchars=\\\{\}}
% Add ',fontsize=\small' for more characters per line
\usepackage{framed}
\definecolor{shadecolor}{RGB}{248,248,248}
\newenvironment{Shaded}{\begin{snugshade}}{\end{snugshade}}
\newcommand{\AlertTok}[1]{\textcolor[rgb]{0.94,0.16,0.16}{#1}}
\newcommand{\AnnotationTok}[1]{\textcolor[rgb]{0.56,0.35,0.01}{\textbf{\textit{#1}}}}
\newcommand{\AttributeTok}[1]{\textcolor[rgb]{0.77,0.63,0.00}{#1}}
\newcommand{\BaseNTok}[1]{\textcolor[rgb]{0.00,0.00,0.81}{#1}}
\newcommand{\BuiltInTok}[1]{#1}
\newcommand{\CharTok}[1]{\textcolor[rgb]{0.31,0.60,0.02}{#1}}
\newcommand{\CommentTok}[1]{\textcolor[rgb]{0.56,0.35,0.01}{\textit{#1}}}
\newcommand{\CommentVarTok}[1]{\textcolor[rgb]{0.56,0.35,0.01}{\textbf{\textit{#1}}}}
\newcommand{\ConstantTok}[1]{\textcolor[rgb]{0.00,0.00,0.00}{#1}}
\newcommand{\ControlFlowTok}[1]{\textcolor[rgb]{0.13,0.29,0.53}{\textbf{#1}}}
\newcommand{\DataTypeTok}[1]{\textcolor[rgb]{0.13,0.29,0.53}{#1}}
\newcommand{\DecValTok}[1]{\textcolor[rgb]{0.00,0.00,0.81}{#1}}
\newcommand{\DocumentationTok}[1]{\textcolor[rgb]{0.56,0.35,0.01}{\textbf{\textit{#1}}}}
\newcommand{\ErrorTok}[1]{\textcolor[rgb]{0.64,0.00,0.00}{\textbf{#1}}}
\newcommand{\ExtensionTok}[1]{#1}
\newcommand{\FloatTok}[1]{\textcolor[rgb]{0.00,0.00,0.81}{#1}}
\newcommand{\FunctionTok}[1]{\textcolor[rgb]{0.00,0.00,0.00}{#1}}
\newcommand{\ImportTok}[1]{#1}
\newcommand{\InformationTok}[1]{\textcolor[rgb]{0.56,0.35,0.01}{\textbf{\textit{#1}}}}
\newcommand{\KeywordTok}[1]{\textcolor[rgb]{0.13,0.29,0.53}{\textbf{#1}}}
\newcommand{\NormalTok}[1]{#1}
\newcommand{\OperatorTok}[1]{\textcolor[rgb]{0.81,0.36,0.00}{\textbf{#1}}}
\newcommand{\OtherTok}[1]{\textcolor[rgb]{0.56,0.35,0.01}{#1}}
\newcommand{\PreprocessorTok}[1]{\textcolor[rgb]{0.56,0.35,0.01}{\textit{#1}}}
\newcommand{\RegionMarkerTok}[1]{#1}
\newcommand{\SpecialCharTok}[1]{\textcolor[rgb]{0.00,0.00,0.00}{#1}}
\newcommand{\SpecialStringTok}[1]{\textcolor[rgb]{0.31,0.60,0.02}{#1}}
\newcommand{\StringTok}[1]{\textcolor[rgb]{0.31,0.60,0.02}{#1}}
\newcommand{\VariableTok}[1]{\textcolor[rgb]{0.00,0.00,0.00}{#1}}
\newcommand{\VerbatimStringTok}[1]{\textcolor[rgb]{0.31,0.60,0.02}{#1}}
\newcommand{\WarningTok}[1]{\textcolor[rgb]{0.56,0.35,0.01}{\textbf{\textit{#1}}}}
\usepackage{graphicx}
\makeatletter
\def\maxwidth{\ifdim\Gin@nat@width>\linewidth\linewidth\else\Gin@nat@width\fi}
\def\maxheight{\ifdim\Gin@nat@height>\textheight\textheight\else\Gin@nat@height\fi}
\makeatother
% Scale images if necessary, so that they will not overflow the page
% margins by default, and it is still possible to overwrite the defaults
% using explicit options in \includegraphics[width, height, ...]{}
\setkeys{Gin}{width=\maxwidth,height=\maxheight,keepaspectratio}
% Set default figure placement to htbp
\makeatletter
\def\fps@figure{htbp}
\makeatother
\setlength{\emergencystretch}{3em} % prevent overfull lines
\providecommand{\tightlist}{%
  \setlength{\itemsep}{0pt}\setlength{\parskip}{0pt}}
\setcounter{secnumdepth}{-\maxdimen} % remove section numbering
\usepackage{booktabs}
\usepackage{longtable}
\usepackage{array}
\usepackage{multirow}
\usepackage{wrapfig}
\usepackage{float}
\usepackage{colortbl}
\usepackage{pdflscape}
\usepackage{tabu}
\usepackage{threeparttable}
\usepackage{threeparttablex}
\usepackage[normalem]{ulem}
\usepackage{makecell}
\usepackage{xcolor}
\ifLuaTeX
  \usepackage{selnolig}  % disable illegal ligatures
\fi

\title{Pluralsight Model Building Take-Home}
\author{William Rinauto}
\date{2022-03-23}

\begin{document}
\maketitle

\begin{Shaded}
\begin{Highlighting}[]
\FunctionTok{library}\NormalTok{(tidyverse)}
\FunctionTok{library}\NormalTok{(h2o)}
\FunctionTok{library}\NormalTok{(vtable)}
\FunctionTok{library}\NormalTok{(scales)}
\FunctionTok{library}\NormalTok{(ggcorrplot)}
\FunctionTok{library}\NormalTok{(kableExtra)}
\FunctionTok{library}\NormalTok{(rpart)}
\FunctionTok{library}\NormalTok{(rpart.plot)}
\end{Highlighting}
\end{Shaded}

\begin{Shaded}
\begin{Highlighting}[]
\NormalTok{training\_dat }\OtherTok{\textless{}{-}} \FunctionTok{read.csv}\NormalTok{(}\StringTok{\textquotesingle{}recruiting\_zeta{-}disease\_training{-}data\_take{-}home{-}challenge {-} 2021\_zeta{-}disease\_training{-}data\_take{-}home{-}challenge.csv\textquotesingle{}}\NormalTok{)}
\NormalTok{predict\_these }\OtherTok{\textless{}{-}} \FunctionTok{read.csv}\NormalTok{(}\StringTok{\textquotesingle{}recruiting\_zeta{-}disease\_prediction{-}data\_take{-}home{-}challenge {-} 2021{-}01{-}21\_zeta{-}disease\_prediction{-}data\_take{-}home{-}challenge.csv\textquotesingle{}}\NormalTok{)}

\CommentTok{\#Check for duplicate rows.}
\CommentTok{\#Even though there is no person identifier in this data, I am going to assume that if }
\CommentTok{\#data values are the same for each data field in two or more rows, then that is duplicate data}
\CommentTok{\#(meaning the same person has been included twice in the data). }
\CommentTok{\#I think this is a fair assumption because it is highly improbable that two individuals would match }
\CommentTok{\#across all of these features. }

\CommentTok{\#How many rows will be dropped: }
\FunctionTok{nrow}\NormalTok{(training\_dat) }\SpecialCharTok{{-}} \FunctionTok{nrow}\NormalTok{(}\FunctionTok{distinct}\NormalTok{(training\_dat))}
\end{Highlighting}
\end{Shaded}

{[}1{]} 5

\begin{Shaded}
\begin{Highlighting}[]
\CommentTok{\#Check to make sure that a person hasn\textquotesingle{}t been recorded twice with two different values for zeta\_disease}
\CommentTok{\#If this matches return value of above line of code, then I\textquotesingle{}ll know they haven\textquotesingle{}t}

\FunctionTok{nrow}\NormalTok{(training\_dat }\SpecialCharTok{\%\textgreater{}\%} \FunctionTok{select}\NormalTok{(}\SpecialCharTok{{-}}\NormalTok{zeta\_disease)) }\SpecialCharTok{{-}} \FunctionTok{nrow}\NormalTok{(}\FunctionTok{distinct}\NormalTok{(training\_dat }\SpecialCharTok{\%\textgreater{}\%} \FunctionTok{select}\NormalTok{(}\SpecialCharTok{{-}}\NormalTok{zeta\_disease)))}
\end{Highlighting}
\end{Shaded}

{[}1{]} 5

\begin{Shaded}
\begin{Highlighting}[]
\CommentTok{\#Drop the 5 duplicate rows:}
\NormalTok{training\_dat }\OtherTok{\textless{}{-}} \FunctionTok{distinct}\NormalTok{(training\_dat)}

\CommentTok{\#What proportion of training data is zeta positive? }
\FunctionTok{mean}\NormalTok{(training\_dat}\SpecialCharTok{$}\NormalTok{zeta\_disease)}
\end{Highlighting}
\end{Shaded}

{[}1{]} 0.3496855

\begin{Shaded}
\begin{Highlighting}[]
\CommentTok{\#Define function to count NA values in each column of some dataframe}
\NormalTok{na\_count }\OtherTok{\textless{}{-}} \ControlFlowTok{function}\NormalTok{(df) }\FunctionTok{sapply}\NormalTok{(df, }\ControlFlowTok{function}\NormalTok{(c) }\FunctionTok{sum}\NormalTok{(}\FunctionTok{is.na}\NormalTok{(c)))}

\CommentTok{\#A return value of true here indicates that there is no missing data in training data}
\FunctionTok{all}\NormalTok{(}\FunctionTok{na\_count}\NormalTok{(training\_dat) }\SpecialCharTok{==} \DecValTok{0}\NormalTok{)}
\end{Highlighting}
\end{Shaded}

{[}1{]} TRUE

\begin{Shaded}
\begin{Highlighting}[]
\CommentTok{\#A return value of true here indicates that there is no missing data in testing data}
\FunctionTok{all}\NormalTok{(}\FunctionTok{na\_count}\NormalTok{(predict\_these) }\SpecialCharTok{==} \DecValTok{0}\NormalTok{)}
\end{Highlighting}
\end{Shaded}

{[}1{]} FALSE

\begin{Shaded}
\begin{Highlighting}[]
\CommentTok{\#Returns False so I will investigate further:}
\FunctionTok{na\_count}\NormalTok{(predict\_these)}
\end{Highlighting}
\end{Shaded}

\begin{verbatim}
           age             weight                bmi     blood_pressure 
             0                  0                  0                  0 
  insulin_test  liver_stress_test cardio_stress_test      years_smoking 
             0                  0                  0                  0 
  zeta_disease 
            20 
\end{verbatim}

\begin{Shaded}
\begin{Highlighting}[]
\CommentTok{\#In this case, I can see that the only "missing" data is zeta\_disease, which makes sense because it hasn\textquotesingle{}t been predicted yet}


\CommentTok{\#Summary statistics table to get a quick sense of data distributions and also sanity check }
\CommentTok{\#for outliers in mins and maximums (negative weight or age for example would be an indicator of bad data)}

\NormalTok{clean\_names }\OtherTok{\textless{}{-}} \ControlFlowTok{function}\NormalTok{(x) \{}
  \FunctionTok{names}\NormalTok{(x) }\OtherTok{\textless{}{-}} \FunctionTok{str\_replace\_all}\NormalTok{(}\FunctionTok{names}\NormalTok{(x), }\StringTok{"\_"}\NormalTok{, }\StringTok{"."}\NormalTok{)}
  \FunctionTok{return}\NormalTok{(x)}
\NormalTok{\}}

\FunctionTok{sumtable}\NormalTok{(training\_dat }\SpecialCharTok{\%\textgreater{}\%} \FunctionTok{clean\_names}\NormalTok{())}
\end{Highlighting}
\end{Shaded}

\begin{table}

\caption{\label{tab:dataLoad}Summary Statistics}
\centering
\begin{tabular}[t]{llllllll}
\toprule
Variable & N & Mean & Std. Dev. & Min & Pctl. 25 & Pctl. 75 & Max\\
\midrule
age & 795 & 30.636 & 12.861 & 18 & 21 & 38 & 109\\
weight & 795 & 172.376 & 31.687 & 94 & 150 & 192 & 308\\
bmi & 795 & 32.231 & 8.565 & 0 & 27.3 & 36.6 & 86.1\\
blood.pressure & 795 & 69.567 & 19.922 & 0 & 62 & 80 & 157\\
insulin.test & 795 & 85.906 & 126.687 & 0 & 0 & 130 & 1077\\
\addlinespace
liver.stress.test & 795 & 0.544 & 0.348 & 0.141 & 0.308 & 0.7 & 3.481\\
cardio.stress.test & 795 & 43.067 & 30.495 & 0 & 0 & 62 & 214\\
years.smoking & 795 & 4.057 & 4.177 & 0 & 1 & 6 & 40\\
zeta.disease & 795 & 0.35 & 0.477 & 0 & 0 & 1 & 1\\
\bottomrule
\end{tabular}
\end{table}

\begin{Shaded}
\begin{Highlighting}[]
\CommentTok{\#Also want to see summary table of the testing data}
\FunctionTok{sumtable}\NormalTok{(predict\_these }\SpecialCharTok{\%\textgreater{}\%} \FunctionTok{clean\_names}\NormalTok{())}
\end{Highlighting}
\end{Shaded}

\begin{table}

\caption{\label{tab:dataLoad}Summary Statistics}
\centering
\begin{tabular}[t]{llllllll}
\toprule
Variable & N & Mean & Std. Dev. & Min & Pctl. 25 & Pctl. 75 & Max\\
\midrule
age & 20 & 34.75 & 11.511 & 19 & 26.25 & 44.25 & 60\\
weight & 20 & 178.8 & 27.935 & 120 & 153.25 & 197.75 & 216\\
bmi & 20 & 34.48 & 6.629 & 25.8 & 30.25 & 37.6 & 50.7\\
blood.pressure & 20 & 78.5 & 14.006 & 59 & 69.75 & 89.25 & 108\\
insulin.test & 20 & 145.05 & 75.964 & 50 & 76.25 & 167.75 & 362\\
\addlinespace
liver.stress.test & 20 & 1.57 & 0.23 & 1.25 & 1.412 & 1.738 & 2.051\\
cardio.stress.test & 20 & 61.95 & 9.703 & 43 & 55.75 & 68 & 83\\
years.smoking & 20 & 6.05 & 3.471 & 2 & 3 & 7.5 & 13\\
zeta.disease & 0 &  &  &  &  &  & \\
... No & 0 & NaN% &  &  &  &  & \\
\addlinespace
... Yes & 0 & NaN% &  &  &  &  & \\
\bottomrule
\end{tabular}
\end{table}

\begin{Shaded}
\begin{Highlighting}[]
\CommentTok{\#A few things stick out in the summary table for the training data. }
\CommentTok{\#There are 5 fields (ignoring zeta\_disease) that have minimum values of 0. }
\CommentTok{\#Two of these stick out to me, because I would think that there is something problematic about having a 0 value here:}
\CommentTok{\#bmi and blood\_pressure}

\CommentTok{\#Let\textquotesingle{}s investigate further}


\CommentTok{\#How often does 0 occur by data field? }

\CommentTok{\#Define function to count 0\textquotesingle{}s. }
\NormalTok{zero\_count }\OtherTok{\textless{}{-}} \ControlFlowTok{function}\NormalTok{(df) }\FunctionTok{sapply}\NormalTok{(df, }\ControlFlowTok{function}\NormalTok{(c) }\FunctionTok{percent}\NormalTok{(}\FunctionTok{mean}\NormalTok{(c }\SpecialCharTok{==} \DecValTok{0}\NormalTok{)))}
\FunctionTok{zero\_count}\NormalTok{(training\_dat)}
\end{Highlighting}
\end{Shaded}

\begin{verbatim}
           age             weight                bmi     blood_pressure 
          "0%"               "0%"               "1%"               "4%" 
  insulin_test  liver_stress_test cardio_stress_test      years_smoking 
         "47%"               "0%"              "29%"              "14%" 
  zeta_disease 
         "65%" 
\end{verbatim}

\begin{Shaded}
\begin{Highlighting}[]
\CommentTok{\#I can see that 1\% of the data has bmi = 0, and 4\% of the data has blood\_pressure = 0}
\CommentTok{\#(Not as important, but we should probably understand why only 14\% of people in this dataset don\textquotesingle{}t smoke. Is this a non{-}representative sample, or is something going on on mars}
\CommentTok{\#that makes people more inclined to take up smoking?)}

\CommentTok{\#At this point, I am going to make two assumptions moving forward with this project:}

\CommentTok{\#ASSUMPTION 1: A person\textquotesingle{}s BMI cannot be 0. According to the data dictionary, bmi = weight/height. Since weight definitely can\textquotesingle{}t be 0}
\CommentTok{\#(and a weight of 0 would have showed up in the data, assuming that it is the same instance of weight measured here as was used to calculate}
\CommentTok{\#bmi), bmi can\textquotesingle{}t be 0.}

\CommentTok{\#ASSUMPTION 2: A person\textquotesingle{}s blood\_pressure CANNOT be 0. I believe this means that the heart has stopped. I\textquotesingle{}m assuming no dead people had data collected here}

\CommentTok{\#Assumptions 1 and 2 require me to take action to clean the data:}

\CommentTok{\#Some possible options for each field: }
\CommentTok{\#1. Mean imputation}
\CommentTok{\#2. Group mean imputation}
\CommentTok{\#3. KNN imputation}
\CommentTok{\#4. Delete bad data}

\CommentTok{\#To keep things simple, I will use mean imputation and replace all 0 values for blood\_pressure and bmi with the mean of all non{-}zero values for each}
\NormalTok{non\_zer\_bp }\OtherTok{\textless{}{-}}\NormalTok{ training\_dat[training\_dat}\SpecialCharTok{$}\NormalTok{blood\_pressure }\SpecialCharTok{\textgreater{}} \DecValTok{0}\NormalTok{,]}
\NormalTok{avg\_bp }\OtherTok{\textless{}{-}} \FunctionTok{mean}\NormalTok{(non\_zer\_bp}\SpecialCharTok{$}\NormalTok{blood\_pressure)}

\NormalTok{non\_zer\_bmi }\OtherTok{\textless{}{-}}\NormalTok{ training\_dat[training\_dat}\SpecialCharTok{$}\NormalTok{bmi }\SpecialCharTok{\textgreater{}} \DecValTok{0}\NormalTok{,]}
\NormalTok{avg\_bmi }\OtherTok{\textless{}{-}} \FunctionTok{mean}\NormalTok{(non\_zer\_bmi}\SpecialCharTok{$}\NormalTok{bmi)}

\NormalTok{replacements\_vals }\OtherTok{\textless{}{-}} \FunctionTok{tibble}\NormalTok{(}\AttributeTok{blood\_pressure =}\NormalTok{ avg\_bp,}
                            \AttributeTok{bmi =}\NormalTok{ avg\_bmi)}

\CommentTok{\#This will be used in "production" environment to clean data before}
\CommentTok{\#creating predictions}
\FunctionTok{write.csv}\NormalTok{(replacements\_vals, }\StringTok{\textquotesingle{}replacement\_vals.csv\textquotesingle{}}\NormalTok{,}\AttributeTok{row.names =}\NormalTok{F)}

\NormalTok{training\_dat }\OtherTok{\textless{}{-}}\NormalTok{ training\_dat }\SpecialCharTok{\%\textgreater{}\%}
  \CommentTok{\#Replace values of 0 with non{-}zero means}
  \FunctionTok{mutate}\NormalTok{(}\AttributeTok{blood\_pressure =} \FunctionTok{ifelse}\NormalTok{(blood\_pressure }\SpecialCharTok{==} \DecValTok{0}\NormalTok{, avg\_bp, blood\_pressure),}
         \AttributeTok{bmi =} \FunctionTok{ifelse}\NormalTok{(bmi }\SpecialCharTok{==} \DecValTok{0}\NormalTok{, avg\_bmi, bmi))}



\CommentTok{\#see that min blood\_pressure and bmi are no longer 0}
\FunctionTok{min}\NormalTok{(training\_dat}\SpecialCharTok{$}\NormalTok{blood\_pressure)}
\end{Highlighting}
\end{Shaded}

{[}1{]} 24

\begin{Shaded}
\begin{Highlighting}[]
\FunctionTok{min}\NormalTok{(training\_dat}\SpecialCharTok{$}\NormalTok{bmi)}
\end{Highlighting}
\end{Shaded}

{[}1{]} 18.2

\begin{Shaded}
\begin{Highlighting}[]
\CommentTok{\#One more sanity check: is years smoking ever greater than age? If it is, that\textquotesingle{}s another indicator of bad data}
\CommentTok{\#In this case, I will use greater than/equal to in my logical statement rather than strict inequality,}
\CommentTok{\#therefore implying that it is not irrational for a person to have begun smoking}
\CommentTok{\#on mars at 0 years{-}old. I will not presume to understand mars culture.}

\CommentTok{\#A return value of false here indicates a logical contradiction in the data}
\FunctionTok{all}\NormalTok{(training\_dat}\SpecialCharTok{$}\NormalTok{age }\SpecialCharTok{\textgreater{}=}\NormalTok{ training\_dat}\SpecialCharTok{$}\NormalTok{years\_smoking)}
\end{Highlighting}
\end{Shaded}

{[}1{]} FALSE

\begin{Shaded}
\begin{Highlighting}[]
\CommentTok{\#Returns false so I will investigate further:}
\NormalTok{smoker\_anomaly\_training }\OtherTok{\textless{}{-}} \FunctionTok{filter}\NormalTok{(training\_dat, years\_smoking }\SpecialCharTok{\textgreater{}}\NormalTok{ age)}
\FunctionTok{head}\NormalTok{(smoker\_anomaly\_training)}
\end{Highlighting}
\end{Shaded}

age weight bmi blood\_pressure insulin\_test liver\_stress\_test 1 19
153 19.4 80 82 0.5538 2 19 158 25.3 62 278 0.9438 cardio\_stress\_test
years\_smoking zeta\_disease 1 41 22 0 2 40 38 0

\begin{Shaded}
\begin{Highlighting}[]
\CommentTok{\#In this case, we can see that two individuals labeled as 19 years old have supposedly been smoking for longer than they\textquotesingle{}ve been alive}
\CommentTok{\#At this point, I have to clean up the data. I can do so by selecting from the four options listed above. }

\CommentTok{\#This situation is a little different than the blood pressure and bmi situations, though, in that I am finding a contradiction in the data by comparing two}
\CommentTok{\#data fields to each other, rather than making a logical assumption about a single data field in isolation. The implication is that I don\textquotesingle{}t know }
\CommentTok{\#which of these two data fields is the incorrect one for these two observations, age or years\_smoking. Therefore, a decision to replace bad data values by imputation would}
\CommentTok{\#be arbitrary: I have no way of saying that age must be replaced and years\_smoking kept, or vice versa. Therefore, I will delete these rows of data entirely. }
\CommentTok{\#Since this is only two rows of data that account for 0.25\% of total observations in this dataset, it should be fairly inconsequential to remove them.}

\CommentTok{\#Delete contradictory data:}
\NormalTok{training\_dat }\OtherTok{\textless{}{-}} \FunctionTok{filter}\NormalTok{(training\_dat, age }\SpecialCharTok{\textgreater{}=}\NormalTok{ years\_smoking)}


\CommentTok{\#Summary table after cleaning up data:}
\FunctionTok{sumtable}\NormalTok{(training\_dat)}
\end{Highlighting}
\end{Shaded}

\begin{table}

\caption{\label{tab:dataLoad}Summary Statistics}
\centering
\begin{tabular}[t]{llllllll}
\toprule
Variable & N & Mean & Std. Dev. & Min & Pctl. 25 & Pctl. 75 & Max\\
\midrule
age & 793 & 30.666 & 12.864 & 18 & 21 & 38 & 109\\
weight & 793 & 172.419 & 31.715 & 94 & 150 & 192 & 308\\
bmi & 793 & 32.709 & 7.656 & 18.2 & 27.6 & 36.6 & 86.1\\
blood_pressure & 793 & 72.776 & 13.189 & 24 & 64 & 80 & 157\\
insulin_test & 793 & 85.668 & 126.663 & 0 & 0 & 130 & 1077\\
\addlinespace
liver_stress_test & 793 & 0.543 & 0.348 & 0.141 & 0.308 & 0.7 & 3.481\\
cardio_stress_test & 793 & 43.073 & 30.534 & 0 & 0 & 62 & 214\\
years_smoking & 793 & 3.991 & 3.953 & 0 & 1 & 6 & 40\\
zeta_disease & 793 & 0.351 & 0.477 & 0 & 0 & 1 & 1\\
\bottomrule
\end{tabular}
\end{table}

\begin{Shaded}
\begin{Highlighting}[]
\CommentTok{\#Now that I am done cleaning the training data, I am going to }
\CommentTok{\#search for relationships in the data}

\NormalTok{cor\_dat }\OtherTok{\textless{}{-}}\NormalTok{ training\_dat}
\CommentTok{\# names(cor\_dat) \textless{}{-} knitr:::escape\_latex(names(cor\_dat))}

\NormalTok{cors }\OtherTok{\textless{}{-}} \FunctionTok{cor}\NormalTok{(cor\_dat)}
\NormalTok{cors }\OtherTok{\textless{}{-}}\NormalTok{ cors[,}\FunctionTok{order}\NormalTok{(cors[}\StringTok{\textquotesingle{}zeta\_disease\textquotesingle{}}\NormalTok{,], }\AttributeTok{decreasing =}\NormalTok{ F)]}
\NormalTok{cor\_plot }\OtherTok{\textless{}{-}} \FunctionTok{ggcorrplot}\NormalTok{(cors,}
                       \CommentTok{\# colors = c(min(cors), 0, max(cors[cors!=1])),}
                       \AttributeTok{type =} \StringTok{\textquotesingle{}lower\textquotesingle{}}\NormalTok{,}
                       \AttributeTok{lab =}\NormalTok{ T) }\SpecialCharTok{+}
  \FunctionTok{scale\_fill\_gradient2}\NormalTok{(}\AttributeTok{limit =} \FunctionTok{c}\NormalTok{(}\FunctionTok{min}\NormalTok{(cors),}\FunctionTok{max}\NormalTok{(cors[cors}\SpecialCharTok{!=}\DecValTok{1}\NormalTok{])), }\AttributeTok{low =} \StringTok{"blue"}\NormalTok{, }\AttributeTok{high =}  \StringTok{"forestgreen"}\NormalTok{, }\AttributeTok{mid =} \StringTok{"white"}\NormalTok{, }\AttributeTok{midpoint =} \DecValTok{0}\NormalTok{) }\SpecialCharTok{+}
  \FunctionTok{ggtitle}\NormalTok{(}\StringTok{\textquotesingle{}Training Data Correlations}\SpecialCharTok{\textbackslash{}n}\StringTok{(Sorted desc on zeta\_disease correlation)\textquotesingle{}}\NormalTok{)}

\FunctionTok{print}\NormalTok{(cor\_plot)}
\end{Highlighting}
\end{Shaded}

\includegraphics{model_files/figure-latex/dataLoad-1.pdf}

\begin{Shaded}
\begin{Highlighting}[]
\CommentTok{\#Based on correlations alone, weight immediately sticks out as potentially predictive. Cardio stress test seems like it will be}
\CommentTok{\#the least predictive}


\CommentTok{\#Another angle I\textquotesingle{}d like to see on the data is average of each candidate predictor based on whether or not the person is}
\CommentTok{\#zeta positive}

\CommentTok{\#This shows an average profile of a zeta positive individual compared to a non zeta positive individual}

\NormalTok{zeta\_means }\OtherTok{\textless{}{-}}\NormalTok{ training\_dat }\SpecialCharTok{\%\textgreater{}\%}
  \FunctionTok{group\_by}\NormalTok{(zeta\_disease) }\SpecialCharTok{\%\textgreater{}\%}
  \FunctionTok{summarise\_all}\NormalTok{(mean) }\SpecialCharTok{\%\textgreater{}\%} 
  \FunctionTok{left\_join}\NormalTok{(training\_dat }\SpecialCharTok{\%\textgreater{}\%} \FunctionTok{count}\NormalTok{(zeta\_disease,}\AttributeTok{name =} \StringTok{\textquotesingle{}Count\textquotesingle{}}\NormalTok{)) }\SpecialCharTok{\%\textgreater{}\%}
  \FunctionTok{arrange}\NormalTok{(}\FunctionTok{desc}\NormalTok{(zeta\_disease))}


\CommentTok{\#Chop off some decimal places}
\NormalTok{rnd }\OtherTok{\textless{}{-}} \ControlFlowTok{function}\NormalTok{(x) }\FunctionTok{round}\NormalTok{(x,}\DecValTok{2}\NormalTok{)}

\CommentTok{\#mutate\_all is a quick way to apply a function to every column in a dataframe}
\NormalTok{zeta\_means }\OtherTok{\textless{}{-}} \FunctionTok{mutate\_all}\NormalTok{(zeta\_means, rnd)}

\FunctionTok{kable}\NormalTok{(zeta\_means }\SpecialCharTok{\%\textgreater{}\%} \FunctionTok{clean\_names}\NormalTok{(), }\AttributeTok{format =} \StringTok{\textquotesingle{}latex\textquotesingle{}}\NormalTok{) }
\end{Highlighting}
\end{Shaded}

\begin{tabular}{r|r|r|r|r|r|r|r|r|r}
\hline
zeta.disease & age & weight & bmi & blood.pressure & insulin.test & liver.stress.test & cardio.stress.test & years.smoking & Count\\
\hline
1 & 34.40 & 192.92 & 35.62 & 74.98 & 107.56 & 0.63 & 44.57 & 5.14 & 278\\
\hline
0 & 28.65 & 161.35 & 31.14 & 71.59 & 73.85 & 0.50 & 42.26 & 3.37 & 515\\
\hline
\end{tabular}

\begin{Shaded}
\begin{Highlighting}[]
\CommentTok{\# \%\textgreater{}\%}
\CommentTok{\#   kable\_styling(bootstrap\_options = "bordered",}
\CommentTok{\#                 full\_width = FALSE)}

\CommentTok{\#Want to also see how various candidate predictors are distributed}

\CommentTok{\#Create function that takes dataframe and column name as input, and outputs density plot}
\NormalTok{plot\_dense }\OtherTok{\textless{}{-}} \ControlFlowTok{function}\NormalTok{(dat, col) \{}
  
  \FunctionTok{ggplot}\NormalTok{(dat, }\FunctionTok{aes\_string}\NormalTok{(}\AttributeTok{x =}\NormalTok{ col)) }\SpecialCharTok{+} 
    \FunctionTok{geom\_density}\NormalTok{(}\AttributeTok{lwd =}\NormalTok{ .}\DecValTok{8}\NormalTok{, }\AttributeTok{fill =} \StringTok{\textquotesingle{}blue\textquotesingle{}}\NormalTok{,}\AttributeTok{alpha =}\NormalTok{ .}\DecValTok{2}\NormalTok{) }\SpecialCharTok{+}
    \FunctionTok{ylab}\NormalTok{(}\StringTok{\textquotesingle{}Density\textquotesingle{}}\NormalTok{) }\SpecialCharTok{+}
    \FunctionTok{ggtitle}\NormalTok{(c) }\SpecialCharTok{+}
    \FunctionTok{xlab}\NormalTok{(}\FunctionTok{paste0}\NormalTok{(c, }\StringTok{"}\SpecialCharTok{\textbackslash{}n\textbackslash{}n\textbackslash{}n}\StringTok{"}\NormalTok{))}
  
  
\NormalTok{\}}

\NormalTok{cols }\OtherTok{\textless{}{-}} \FunctionTok{names}\NormalTok{(training\_dat)[}\SpecialCharTok{!}\FunctionTok{names}\NormalTok{(training\_dat) }\SpecialCharTok{==} \StringTok{\textquotesingle{}zeta\_disease\textquotesingle{}}\NormalTok{]}

\ControlFlowTok{for}\NormalTok{(c }\ControlFlowTok{in}\NormalTok{ cols) \{}
  
\NormalTok{  p }\OtherTok{\textless{}{-}} \FunctionTok{plot\_dense}\NormalTok{(training\_dat, c)}
  \FunctionTok{print}\NormalTok{(p)}
\NormalTok{\}}
\end{Highlighting}
\end{Shaded}

\includegraphics{model_files/figure-latex/dataLoad-2.pdf}
\includegraphics{model_files/figure-latex/dataLoad-3.pdf}
\includegraphics{model_files/figure-latex/dataLoad-4.pdf}
\includegraphics{model_files/figure-latex/dataLoad-5.pdf}
\includegraphics{model_files/figure-latex/dataLoad-6.pdf}
\includegraphics{model_files/figure-latex/dataLoad-7.pdf}
\includegraphics{model_files/figure-latex/dataLoad-8.pdf}
\includegraphics{model_files/figure-latex/dataLoad-9.pdf}

\begin{Shaded}
\begin{Highlighting}[]
\CommentTok{\#Also interested in seeing each variable plotted against zeta\_disease in a scatterplot}
\CommentTok{\#I will add a smoothing line to give some indicacator of the relationship between each}
\CommentTok{\#variable and zeta\_disease. It\textquotesingle{}s important to note that outliers can have a major impact}
\CommentTok{\#on the visual interpretation of the smoothing line, so a wider "shadow" around the line essentially }
\CommentTok{\#indicates less trustworthiness in the shape of the line within that region of data.}

\CommentTok{\#Create scatterplot function}
\NormalTok{plot\_scatter }\OtherTok{\textless{}{-}} \ControlFlowTok{function}\NormalTok{(dat1, dat2, col) \{}
  
  \FunctionTok{ggplot}\NormalTok{(dat1, }\FunctionTok{aes\_string}\NormalTok{(}\AttributeTok{x =}\NormalTok{ col, }\AttributeTok{y =} \StringTok{\textquotesingle{}zeta\_disease\textquotesingle{}}\NormalTok{, }\AttributeTok{color =} \StringTok{\textquotesingle{}zeta\_disease\textquotesingle{}}\NormalTok{)) }\SpecialCharTok{+}
    \FunctionTok{stat\_smooth}\NormalTok{(}\AttributeTok{method=}\StringTok{"glm"}\NormalTok{, }\AttributeTok{color=}\StringTok{"black"}\NormalTok{, }\AttributeTok{se=}\NormalTok{T,}
                \AttributeTok{method.args =} \FunctionTok{list}\NormalTok{(}\AttributeTok{family=}\NormalTok{binomial)) }\SpecialCharTok{+}     \FunctionTok{geom\_point}\NormalTok{() }\SpecialCharTok{+}
    \CommentTok{\# geom\_vline(data = dat2, aes\_string(xintercept = col, color = \textquotesingle{}zeta\_disease\textquotesingle{}), lwd = 1) +}
    \FunctionTok{xlab}\NormalTok{(}\FunctionTok{paste0}\NormalTok{(c, }\StringTok{"}\SpecialCharTok{\textbackslash{}n\textbackslash{}n\textbackslash{}n}\StringTok{"}\NormalTok{)) }\SpecialCharTok{+}
    \FunctionTok{ggtitle}\NormalTok{(c) }\SpecialCharTok{+}
    \FunctionTok{scale\_y\_continuous}\NormalTok{(}\AttributeTok{breaks =} \FunctionTok{c}\NormalTok{(}\DecValTok{0}\NormalTok{,}\DecValTok{1}\NormalTok{)) }\SpecialCharTok{+}
    \FunctionTok{theme}\NormalTok{(}\AttributeTok{legend.position =} \StringTok{\textquotesingle{}none\textquotesingle{}}\NormalTok{) }\SpecialCharTok{+}
    \FunctionTok{ylab}\NormalTok{(}\StringTok{\textquotesingle{}Zeta Status\textquotesingle{}}\NormalTok{)}
  
  
  
\NormalTok{\}}



\ControlFlowTok{for}\NormalTok{(c }\ControlFlowTok{in}\NormalTok{ cols) \{}
  
\NormalTok{  p }\OtherTok{\textless{}{-}} \FunctionTok{plot\_scatter}\NormalTok{(training\_dat, zeta\_means, c)}
  \FunctionTok{print}\NormalTok{(p)}
\NormalTok{\}}
\end{Highlighting}
\end{Shaded}

\includegraphics{model_files/figure-latex/dataLoad-10.pdf}
\includegraphics{model_files/figure-latex/dataLoad-11.pdf}
\includegraphics{model_files/figure-latex/dataLoad-12.pdf}
\includegraphics{model_files/figure-latex/dataLoad-13.pdf}
\includegraphics{model_files/figure-latex/dataLoad-14.pdf}
\includegraphics{model_files/figure-latex/dataLoad-15.pdf}
\includegraphics{model_files/figure-latex/dataLoad-16.pdf}
\includegraphics{model_files/figure-latex/dataLoad-17.pdf}

\begin{Shaded}
\begin{Highlighting}[]
\CommentTok{\#Tree{-}{-}{-}{-}{-}{-}}


\CommentTok{\#Another thing I want to see to get a very simple idea of how these features influence the outcome and interact with}
\CommentTok{\#one another is a decision tree}

\CommentTok{\#Indicate in training data that zeta\_disease is categorical}
\NormalTok{training\_dat }\OtherTok{\textless{}{-}} \FunctionTok{mutate}\NormalTok{(training\_dat, }\AttributeTok{zeta\_disease =} \FunctionTok{as.factor}\NormalTok{(zeta\_disease))}

\CommentTok{\#I am going to shorten the variable names just so the plot prints tidier}
\NormalTok{tree\_dat }\OtherTok{\textless{}{-}}\NormalTok{ training\_dat}

\NormalTok{shorten }\OtherTok{\textless{}{-}} \ControlFlowTok{function}\NormalTok{(x) }\FunctionTok{substr}\NormalTok{(x, }\DecValTok{1}\NormalTok{,}\DecValTok{5}\NormalTok{)}
\FunctionTok{names}\NormalTok{(tree\_dat) }\OtherTok{\textless{}{-}} \FunctionTok{sapply}\NormalTok{(}\FunctionTok{names}\NormalTok{(tree\_dat), shorten)}

\NormalTok{simple\_tree }\OtherTok{\textless{}{-}} \FunctionTok{rpart}\NormalTok{(zeta\_ }\SpecialCharTok{\textasciitilde{}}\NormalTok{ ., }\AttributeTok{data =}\NormalTok{ tree\_dat,}
                     \AttributeTok{control =} \FunctionTok{rpart.control}\NormalTok{(}\AttributeTok{maxdepth =} \DecValTok{4}\NormalTok{))}

\CommentTok{\#Forcing the tree to make only 4 splits at most indicates that weight, }
\CommentTok{\#BMI, and age will likely be important in a final model predicting zeta\_disease}

\FunctionTok{rpart.plot}\NormalTok{(simple\_tree, }\AttributeTok{cex =} \DecValTok{1}\NormalTok{, }\AttributeTok{extra =} \DecValTok{2}\NormalTok{)}
\end{Highlighting}
\end{Shaded}

\includegraphics{model_files/figure-latex/dataLoad-18.pdf}

\begin{Shaded}
\begin{Highlighting}[]
\CommentTok{\#What I\textquotesingle{}m seeing in the tree corroborates some of what I was seeing in the correlation table and plot:}
\CommentTok{\#weight as the root node and the field with the highest importance measure is consistent with it having}
\CommentTok{\#the highest correlation with the target variable. cardio\_stress\_test seems to be inconsequential based on both}
\CommentTok{\#correlation with the target as well as importance in the tree}
\end{Highlighting}
\end{Shaded}

\begin{Shaded}
\begin{Highlighting}[]
\CommentTok{\#Model{-}{-}{-}{-}}

\CommentTok{\#Now that I have cleaned the data and explored some of the relationships, I will }
\CommentTok{\#build a few different classification models and select the one that is most successful}


\CommentTok{\#The first thing I\textquotesingle{}m going to do is drop cardio\_stress\_test from the training data because it seems to be unimportant based }
\CommentTok{\#on the preliminary analysis}

\NormalTok{training\_dat }\OtherTok{\textless{}{-}} \FunctionTok{select}\NormalTok{(training\_dat, }\SpecialCharTok{{-}}\NormalTok{cardio\_stress\_test)}

\CommentTok{\#I also want to add a few features to the data. I could have}
\CommentTok{\#added these in the data exploration step, but I wanted to avoid cluttering}
\CommentTok{\#the rmarkdown document too much}

\CommentTok{\# square \textless{}{-} function(x) x\^{}2}
\CommentTok{\# }
\CommentTok{\# feature\_funs \textless{}{-} list(sqrt = sqrt, square = square)}


\CommentTok{\#mutate\_at to target every variable that is not zeta\_disease}
\CommentTok{\#training\_dat \textless{}{-} mutate\_at(training\_dat, vars(!matches(\textquotesingle{}zeta\_disease\textquotesingle{})), feature\_funs)}


\CommentTok{\#Actually, after creating features this way and running the models, I am seeing that they add nothing}
\CommentTok{\#To the accuracy of any of the models, so I will comment out the prior couple lines of code}

\CommentTok{\#Initialize h2o}
\FunctionTok{h2o.init}\NormalTok{(}\AttributeTok{nthreads =} \SpecialCharTok{{-}}\DecValTok{1}\NormalTok{)}
\end{Highlighting}
\end{Shaded}

H2O is not running yet, starting it now\ldots{}

Note: In case of errors look at the following log files:
C:\Users\willi\AppData\Local\Temp\Rtmp2xCoAk\file13687f55716f/h2o\_willi\_started\_from\_r.out
C:\Users\willi\AppData\Local\Temp\Rtmp2xCoAk\file13687f23e7e/h2o\_willi\_started\_from\_r.err

Starting H2O JVM and connecting: Connection successful!

R is connected to the H2O cluster: H2O cluster uptime: 1 seconds 916
milliseconds H2O cluster timezone: America/New\_York H2O data parsing
timezone: UTC H2O cluster version: 3.36.0.3 H2O cluster version age: 1
month and 6 days\\
H2O cluster name: H2O\_started\_from\_R\_willi\_hwf630 H2O cluster total
nodes: 1 H2O cluster total memory: 15.93 GB H2O cluster total cores: 16
H2O cluster allowed cores: 16 H2O cluster healthy: TRUE H2O Connection
ip: localhost H2O Connection port: 54321 H2O Connection proxy: NA H2O
Internal Security: FALSE R Version: R version 4.1.3 (2022-03-10)

\begin{Shaded}
\begin{Highlighting}[]
\CommentTok{\#Hide progrses bar because it looks bad in rendered document}
\FunctionTok{h2o.no\_progress}\NormalTok{()}

\CommentTok{\#Convert target variable to categorical to avoid accidentally predicting continuous outcome:}
\NormalTok{training\_dat }\OtherTok{\textless{}{-}} \FunctionTok{mutate}\NormalTok{(training\_dat, }\AttributeTok{zeta\_disease =} \FunctionTok{as.factor}\NormalTok{(zeta\_disease))}

\CommentTok{\#Convert training data to h2o object:}
\NormalTok{training\_dat\_h2o  }\OtherTok{\textless{}{-}} \FunctionTok{as.h2o}\NormalTok{(training\_dat)}


\CommentTok{\#this line looks pointless but I had deleted a few lines of code }
\CommentTok{\#where previously it made sense to have this here. Now I\textquotesingle{}m keeping it to avoid}
\CommentTok{\#changing code below}
\NormalTok{train\_h2o }\OtherTok{\textless{}{-}}\NormalTok{ training\_dat\_h2o}


\CommentTok{\#Want to run predictions on final test data as I go along}
\NormalTok{testing\_dat\_h2o }\OtherTok{\textless{}{-}}\NormalTok{ predict\_these }\SpecialCharTok{\%\textgreater{}\%}
  \FunctionTok{select}\NormalTok{(}\SpecialCharTok{{-}}\NormalTok{zeta\_disease) }\SpecialCharTok{\%\textgreater{}\%}
  \FunctionTok{as.h2o}\NormalTok{()}

\CommentTok{\#character vector of candidate regressors:}
\NormalTok{candidate\_regressors }\OtherTok{\textless{}{-}} \FunctionTok{names}\NormalTok{(training\_dat)[}\FunctionTok{names}\NormalTok{(training\_dat) }\SpecialCharTok{!=} \StringTok{\textquotesingle{}zeta\_disease\textquotesingle{}}\NormalTok{]}

\CommentTok{\#target\_variable}
\NormalTok{target }\OtherTok{\textless{}{-}} \StringTok{\textquotesingle{}zeta\_disease\textquotesingle{}}

\CommentTok{\#See all data types to make sure nothing accidentally ended up as categorical somehow (zeta\_disease should show as factor here)}
\FunctionTok{sapply}\NormalTok{(training\_dat, class)}
\end{Highlighting}
\end{Shaded}

\begin{verbatim}
          age            weight               bmi    blood_pressure 
    "integer"         "integer"         "numeric"         "numeric" 
 insulin_test liver_stress_test     years_smoking      zeta_disease 
    "integer"         "numeric"         "integer"          "factor" 
\end{verbatim}

\begin{Shaded}
\begin{Highlighting}[]
\CommentTok{\#For each model, I will use a grid search method to tune hyperparameters}

\CommentTok{\#First test GLM}

\CommentTok{\#It\textquotesingle{}s probably overkill to use elastic net regression but }
\CommentTok{\#if alpha = lamda = 0 is optimal, the grid search will indicate this}
\NormalTok{hyper\_params\_glm }\OtherTok{\textless{}{-}} \FunctionTok{list}\NormalTok{(}
  \CommentTok{\#Controls distribution between ridge and lasso componenets of penalty}
  \CommentTok{\#https://docs.h2o.ai/h2o/latest{-}stable/h2o{-}docs/data{-}science/algo{-}params/alpha.html}
  \AttributeTok{alpha =} \FunctionTok{seq}\NormalTok{(}\AttributeTok{from =} \DecValTok{0}\NormalTok{, }\AttributeTok{to =} \DecValTok{1}\NormalTok{, }\AttributeTok{by =} \FloatTok{0.001}\NormalTok{),}
  
  \CommentTok{\#Amount of regularization (large value here means coefficients shrink closer to 0}
  \CommentTok{\#https://docs.h2o.ai/h2o/latest{-}stable/h2o{-}docs/data{-}science/algo{-}params/lambda.html}
  \AttributeTok{lambda =} \FunctionTok{c}\NormalTok{(.}\DecValTok{00001}\NormalTok{, .}\DecValTok{0001}\NormalTok{, .}\DecValTok{001}\NormalTok{, .}\DecValTok{01}\NormalTok{, .}\DecValTok{1}\NormalTok{, .}\DecValTok{5}\NormalTok{, }\DecValTok{1}\NormalTok{)}
\NormalTok{)}

\CommentTok{\#Number of models to be tested for in absence of RandomDiscrete strategy:}
\FunctionTok{sapply}\NormalTok{(hyper\_params\_glm, length) }\SpecialCharTok{\%\textgreater{}\%} \FunctionTok{prod}\NormalTok{()}
\end{Highlighting}
\end{Shaded}

{[}1{]} 7007

\begin{Shaded}
\begin{Highlighting}[]
\CommentTok{\#Controls how grid search is run (will use same search\_criteria for every model)}
\NormalTok{search\_criteria }\OtherTok{\textless{}{-}} \FunctionTok{list}\NormalTok{(}
  \CommentTok{\#RandomDiscrete grid search samples from parameter space}
  \CommentTok{\#https://docs.h2o.ai/h2o/latest{-}stable/h2o{-}docs/grid{-}search.html}
  \AttributeTok{strategy =} \StringTok{"RandomDiscrete"}\NormalTok{,}
  \AttributeTok{max\_runtime\_secs =} \DecValTok{30}\NormalTok{,}
  \AttributeTok{max\_models =} \DecValTok{200}\NormalTok{,}
  \AttributeTok{stopping\_metric =} \StringTok{"AUC"}\NormalTok{, }
  \AttributeTok{stopping\_tolerance =} \FloatTok{0.00001}\NormalTok{, }
  \AttributeTok{stopping\_rounds =} \DecValTok{5}\NormalTok{, }
  \AttributeTok{seed =} \DecValTok{123}
\NormalTok{)}

\NormalTok{glm\_models }\OtherTok{\textless{}{-}} \FunctionTok{h2o.grid}\NormalTok{(}\AttributeTok{algorithm =} \StringTok{"glm"}\NormalTok{,}
                       \AttributeTok{grid\_id =} \StringTok{"regression"}\NormalTok{,}
                       \AttributeTok{x =}\NormalTok{ candidate\_regressors, }
                       \AttributeTok{y =}\NormalTok{ target, }
                       \AttributeTok{training\_frame =}\NormalTok{ train\_h2o,}
                       \AttributeTok{nfolds =} \DecValTok{10}\NormalTok{, }
                       \AttributeTok{family =} \StringTok{"binomial"}\NormalTok{, }
                       \AttributeTok{hyper\_params =}\NormalTok{ hyper\_params\_glm, }
                       \AttributeTok{search\_criteria =}\NormalTok{ search\_criteria,}
                       \AttributeTok{seed =} \DecValTok{123}\NormalTok{)}

\CommentTok{\#Since h2o.predict will use f1 under the hood to determine classification }
\CommentTok{\#threshold, I will select the model with the highest cv f1 score}
\NormalTok{glm\_sorted }\OtherTok{\textless{}{-}} \FunctionTok{h2o.getGrid}\NormalTok{(}\AttributeTok{grid\_id =} \StringTok{"regression"}\NormalTok{, }\AttributeTok{sort\_by =} \StringTok{"f1"}\NormalTok{, }\AttributeTok{decreasing =} \ConstantTok{TRUE}\NormalTok{)}

\CommentTok{\#Top model when sorted descending on f1}
\NormalTok{glm\_best }\OtherTok{\textless{}{-}} \FunctionTok{h2o.getModel}\NormalTok{(glm\_sorted}\SpecialCharTok{@}\NormalTok{model\_ids[[}\DecValTok{1}\NormalTok{]])}


\CommentTok{\#Useful stackoverflow thread discussing h2o.performance object:}
\CommentTok{\# https://stackoverflow.com/questions/43699454/how{-}to{-}understand{-}the{-}metrics{-}of{-}h2omodelmetrics{-}object{-}through{-}h2o{-}performance}

\CommentTok{\#xval = T below means I am pulling performance data based on cross validation}
\CommentTok{\#testing datasets, not training data. Will do this throughout}
\NormalTok{glm\_performance }\OtherTok{\textless{}{-}} \FunctionTok{h2o.performance}\NormalTok{(glm\_best, }\AttributeTok{xval =}\NormalTok{ T)}
\NormalTok{glm\_f1 }\OtherTok{\textless{}{-}} \FunctionTok{h2o.F1}\NormalTok{(glm\_performance) }\SpecialCharTok{\%\textgreater{}\%}
  \FunctionTok{as.data.frame}\NormalTok{() }\SpecialCharTok{\%\textgreater{}\%}
  \FunctionTok{arrange}\NormalTok{(}\FunctionTok{desc}\NormalTok{(f1)) }

\CommentTok{\#Store f1 value of best performing glm model}
\NormalTok{glm\_f1 }\OtherTok{\textless{}{-}}\NormalTok{ glm\_f1[}\DecValTok{1}\NormalTok{,}\DecValTok{2}\NormalTok{]}

\CommentTok{\#Appy predictions to test data. Will take a look at this later}
\NormalTok{final\_predictions\_glm }\OtherTok{\textless{}{-}} \FunctionTok{h2o.predict}\NormalTok{(glm\_best, testing\_dat\_h2o) }\SpecialCharTok{\%\textgreater{}\%}
  \FunctionTok{as.data.frame}\NormalTok{()}
\FunctionTok{h2o.varimp}\NormalTok{(glm\_best)}
\end{Highlighting}
\end{Shaded}

Variable Importances: variable relative\_importance scaled\_importance
percentage 1 weight 1.095977 1.000000 0.403601 2 bmi 0.565687 0.516149
0.208318 3 years\_smoking 0.355938 0.324768 0.131077 4
liver\_stress\_test 0.310933 0.283704 0.114503 5 age 0.182344 0.166376
0.067149 6 insulin\_test 0.139783 0.127542 0.051476 7 blood\_pressure
0.064837 0.059159 0.023876

\begin{Shaded}
\begin{Highlighting}[]
\CommentTok{\#Next try a random forest. This tree based model will be better if there are }
\CommentTok{\#interactions among variables}

\NormalTok{hyper\_params\_forest }\OtherTok{\textless{}{-}} \FunctionTok{list}\NormalTok{(}
  \CommentTok{\#number of trees}
  \CommentTok{\#https://docs.h2o.ai/h2o/latest{-}stable/h2o{-}docs/data{-}science/algo{-}params/ntrees.html}
  \AttributeTok{ntrees =} \DecValTok{10000}\NormalTok{,  }
  
  \CommentTok{\#how deep tree can go:}
  \CommentTok{\#https://docs.h2o.ai/h2o/latest{-}stable/h2o{-}docs/data{-}science/algo{-}params/max\_depth.html}
  \AttributeTok{max\_depth =} \DecValTok{12}\SpecialCharTok{:}\DecValTok{25}\NormalTok{,}
  
  \CommentTok{\#How much data must be in each bucket to make a split:}
  \CommentTok{\#https://docs.h2o.ai/h2o/latest{-}stable/h2o{-}docs/data{-}science/algo{-}params/min\_rows.html}
  \AttributeTok{min\_rows =} \FunctionTok{seq}\NormalTok{(}\DecValTok{1}\NormalTok{,}\DecValTok{101}\NormalTok{, }\DecValTok{5}\NormalTok{),}
  
  \CommentTok{\#Row sampling rate:}
  \CommentTok{\#https://docs.h2o.ai/h2o/latest{-}stable/h2o{-}docs/data{-}science/algo{-}params/sample\_rate.html}
  \AttributeTok{sample\_rate =} \FunctionTok{seq}\NormalTok{(.}\DecValTok{1}\NormalTok{, }\DecValTok{1}\NormalTok{, }\AttributeTok{by =}\NormalTok{ .}\DecValTok{1}\NormalTok{),}
  
  \CommentTok{\#Number of columns to sample at each node}
  \CommentTok{\#https://docs.h2o.ai/h2o/latest{-}stable/h2o{-}docs/data{-}science/algo{-}params/mtries.html}
  \AttributeTok{mtries =} \FunctionTok{c}\NormalTok{(}\SpecialCharTok{{-}}\DecValTok{1}\NormalTok{,}\DecValTok{1}\SpecialCharTok{:}\DecValTok{7}\NormalTok{)}
\NormalTok{)}


\NormalTok{forest\_models }\OtherTok{\textless{}{-}} \FunctionTok{h2o.grid}\NormalTok{(}\AttributeTok{algorithm =} \StringTok{"randomForest"}\NormalTok{, }
                          \AttributeTok{grid\_id =} \StringTok{"forest"}\NormalTok{, }
                          \AttributeTok{x =}\NormalTok{ candidate\_regressors, }
                          \AttributeTok{y =}\NormalTok{ target, }
                          \AttributeTok{training\_frame =}\NormalTok{ train\_h2o, }
                          \AttributeTok{nfolds =} \DecValTok{10}\NormalTok{, }
                          \AttributeTok{hyper\_params =}\NormalTok{ hyper\_params\_forest, }
                          \AttributeTok{search\_criteria =}\NormalTok{ search\_criteria, }
                          \AttributeTok{seed =} \DecValTok{123}\NormalTok{)}

\NormalTok{forest\_sorted }\OtherTok{\textless{}{-}} \FunctionTok{h2o.getGrid}\NormalTok{(}\AttributeTok{grid\_id =} \StringTok{"forest"}\NormalTok{, }\AttributeTok{sort\_by =} \StringTok{"f1"}\NormalTok{, }\AttributeTok{decreasing =} \ConstantTok{TRUE}\NormalTok{)}

\CommentTok{\#Grab top performing model based on f1}
\NormalTok{forest\_best }\OtherTok{\textless{}{-}} \FunctionTok{h2o.getModel}\NormalTok{(forest\_sorted}\SpecialCharTok{@}\NormalTok{model\_ids[[}\DecValTok{1}\NormalTok{]])}

\NormalTok{forest\_perf }\OtherTok{\textless{}{-}} \FunctionTok{h2o.performance}\NormalTok{(forest\_best, }\AttributeTok{xval =}\NormalTok{ T)}
\NormalTok{forest\_f1 }\OtherTok{\textless{}{-}} \FunctionTok{h2o.F1}\NormalTok{(forest\_perf) }\SpecialCharTok{\%\textgreater{}\%}
  \FunctionTok{as.data.frame}\NormalTok{() }\SpecialCharTok{\%\textgreater{}\%}
  \FunctionTok{arrange}\NormalTok{(}\FunctionTok{desc}\NormalTok{(f1)) }

\CommentTok{\#grab f1 statistic}
\NormalTok{forest\_f1 }\OtherTok{\textless{}{-}}\NormalTok{ forest\_f1[}\DecValTok{1}\NormalTok{,}\DecValTok{2}\NormalTok{]}

\CommentTok{\#Apply predictions to test data}
\NormalTok{final\_predictions\_forest }\OtherTok{\textless{}{-}} \FunctionTok{h2o.predict}\NormalTok{(forest\_best, testing\_dat\_h2o) }\SpecialCharTok{\%\textgreater{}\%}
  \FunctionTok{as.data.frame}\NormalTok{()}


\CommentTok{\#Last I will try gradient boosting:}
\NormalTok{hyper\_params\_gbm }\OtherTok{\textless{}{-}} \FunctionTok{list}\NormalTok{(}\AttributeTok{ntrees =} \DecValTok{10000}\NormalTok{,  }
                         \AttributeTok{max\_depth =} \DecValTok{5}\SpecialCharTok{:}\DecValTok{15}\NormalTok{, }
                         \AttributeTok{min\_rows =} \FunctionTok{c}\NormalTok{(}\DecValTok{15}\NormalTok{, }\DecValTok{20}\NormalTok{,}\DecValTok{30}\NormalTok{, }\DecValTok{50}\NormalTok{,}\DecValTok{100}\NormalTok{),}
                         
                         \CommentTok{\#GBM learn rate (how much to adjust predicted residuals based on new tree)}
                         \CommentTok{\#https://docs.h2o.ai/h2o/latest{-}stable/h2o{-}docs/data{-}science/algo{-}params/learn\_rate.html}
                         \AttributeTok{learn\_rate =} \FunctionTok{c}\NormalTok{(}\FloatTok{0.001}\NormalTok{,}\FloatTok{0.01}\NormalTok{,}\FloatTok{0.1}\NormalTok{, .}\DecValTok{3}\NormalTok{, .}\DecValTok{5}\NormalTok{),  }
                         
                         \CommentTok{\#Change in learn rate each round}
                         \CommentTok{\#https://docs.h2o.ai/h2o/latest{-}stable/h2o{-}docs/data{-}science/algo{-}params/learn\_rate\_annealing.html}
                         \AttributeTok{learn\_rate\_annealing =} \FunctionTok{c}\NormalTok{(}\FloatTok{0.99}\NormalTok{,}\FloatTok{0.999}\NormalTok{,}\DecValTok{1}\NormalTok{),}
                         \AttributeTok{sample\_rate =} \FunctionTok{seq}\NormalTok{(.}\DecValTok{2}\NormalTok{, }\DecValTok{1}\NormalTok{, }\AttributeTok{by =} \DecValTok{1}\NormalTok{),}
                         \AttributeTok{col\_sample\_rate =} \FunctionTok{seq}\NormalTok{(.}\DecValTok{1}\NormalTok{, }\DecValTok{1}\NormalTok{, }\AttributeTok{by =} \DecValTok{1}\NormalTok{)}
                         
\NormalTok{)}


\NormalTok{models\_gbm }\OtherTok{\textless{}{-}} \FunctionTok{h2o.grid}\NormalTok{(}\AttributeTok{algorithm =} \StringTok{"gbm"}\NormalTok{, }\AttributeTok{grid\_id =} \StringTok{"gbm"}\NormalTok{,}
                       \AttributeTok{x =}\NormalTok{ candidate\_regressors, }
                       \AttributeTok{y =}\NormalTok{ target,}
                       \AttributeTok{training\_frame =}\NormalTok{ train\_h2o, }
                       \AttributeTok{nfolds =} \DecValTok{10}\NormalTok{, }
                       \AttributeTok{hyper\_params =}\NormalTok{ hyper\_params\_gbm, }
                       \AttributeTok{search\_criteria =}\NormalTok{ search\_criteria, }
                       \AttributeTok{seed =} \DecValTok{123}\NormalTok{)}

\NormalTok{gbm\_sorted }\OtherTok{\textless{}{-}} \FunctionTok{h2o.getGrid}\NormalTok{(}\AttributeTok{grid\_id =} \StringTok{"gbm"}\NormalTok{, }\AttributeTok{sort\_by =} \StringTok{"f1"}\NormalTok{, }\AttributeTok{decreasing =} \ConstantTok{TRUE}\NormalTok{)}

\CommentTok{\#Best gbm model based on f1 stat}
\NormalTok{gbm\_best }\OtherTok{\textless{}{-}} \FunctionTok{h2o.getModel}\NormalTok{(gbm\_sorted}\SpecialCharTok{@}\NormalTok{model\_ids[[}\DecValTok{1}\NormalTok{]])}

\NormalTok{gbm\_perf }\OtherTok{\textless{}{-}} \FunctionTok{h2o.performance}\NormalTok{(gbm\_best, }\AttributeTok{xval =}\NormalTok{ T)}
\NormalTok{gbm\_f1 }\OtherTok{\textless{}{-}} \FunctionTok{h2o.F1}\NormalTok{(gbm\_perf) }\SpecialCharTok{\%\textgreater{}\%}
  \FunctionTok{as.data.frame}\NormalTok{() }\SpecialCharTok{\%\textgreater{}\%}
  \FunctionTok{arrange}\NormalTok{(}\FunctionTok{desc}\NormalTok{(f1)) }

\CommentTok{\#grab best gbm f1 statistic}
\NormalTok{gbm\_f1 }\OtherTok{\textless{}{-}}\NormalTok{ gbm\_f1[}\DecValTok{1}\NormalTok{,}\DecValTok{2}\NormalTok{]}

\CommentTok{\#Final predictions.. will take a look later}
\NormalTok{final\_predictions\_gbm }\OtherTok{\textless{}{-}} \FunctionTok{h2o.predict}\NormalTok{(gbm\_best, testing\_dat\_h2o) }\SpecialCharTok{\%\textgreater{}\%}
  \FunctionTok{as.data.frame}\NormalTok{()}

\NormalTok{models\_perf\_metric }\OtherTok{\textless{}{-}} \FunctionTok{list}\NormalTok{(}\AttributeTok{glm =}\NormalTok{ glm\_f1, }\AttributeTok{rf =}\NormalTok{ forest\_f1, }\AttributeTok{gbm =}\NormalTok{ gbm\_f1)}
\CommentTok{\#f1 score of each model:}
\NormalTok{models\_perf\_metric}
\end{Highlighting}
\end{Shaded}

\$glm {[}1{]} 0.6872111

\$rf {[}1{]} 0.6970173

\$gbm {[}1{]} 0.696319

\begin{Shaded}
\begin{Highlighting}[]
\CommentTok{\#See how each model predicts on test data}
\NormalTok{dat }\OtherTok{\textless{}{-}} \FunctionTok{bind\_cols}\NormalTok{(}\FunctionTok{list}\NormalTok{(final\_predictions\_glm, final\_predictions\_forest, final\_predictions\_gbm)) }\SpecialCharTok{\%\textgreater{}\%}
  \FunctionTok{select}\NormalTok{(}\FunctionTok{contains}\NormalTok{(}\StringTok{\textquotesingle{}predict\textquotesingle{}}\NormalTok{))}
\FunctionTok{names}\NormalTok{(dat) }\OtherTok{\textless{}{-}} \FunctionTok{c}\NormalTok{(}\StringTok{\textquotesingle{}glm\textquotesingle{}}\NormalTok{,}\StringTok{\textquotesingle{}forest\textquotesingle{}}\NormalTok{,}\StringTok{\textquotesingle{}gbm\textquotesingle{}}\NormalTok{)}
\FunctionTok{head}\NormalTok{(dat, }\DecValTok{20}\NormalTok{)}
\end{Highlighting}
\end{Shaded}

glm forest gbm 1 1 0 0 2 1 1 1 3 0 0 0 4 1 1 1 5 1 1 1 6 0 1 1 7 0 0 0 8
0 0 0 9 1 1 1 10 1 0 0 11 1 1 1 12 1 1 1 13 1 0 0 14 1 1 1 15 1 1 1 16 1
1 1 17 1 1 1 18 1 1 1 19 1 1 1 20 1 1 1

\begin{Shaded}
\begin{Highlighting}[]
\CommentTok{\#proportion of zeta positive predictions by model on test data:}
\NormalTok{convert\_fct\_numeric }\OtherTok{\textless{}{-}} \ControlFlowTok{function}\NormalTok{(x) }\FunctionTok{mean}\NormalTok{(}\FunctionTok{as.numeric}\NormalTok{(}\FunctionTok{as.character}\NormalTok{(x)))}
\FunctionTok{sapply}\NormalTok{(dat, convert\_fct\_numeric)}
\end{Highlighting}
\end{Shaded}

glm forest gbm 0.8 0.7 0.7

\begin{Shaded}
\begin{Highlighting}[]
\CommentTok{\#Based on f1, random forest just barely squeaks out as the winner,}
\CommentTok{\#so I will go with that}

\CommentTok{\#Save model to be ingested by python }
\FunctionTok{h2o.saveModel}\NormalTok{(forest\_best, }\FunctionTok{getwd}\NormalTok{(), }\AttributeTok{filename =} \StringTok{\textquotesingle{}model\textquotesingle{}}\NormalTok{, }\AttributeTok{force =}\NormalTok{ T)}
\end{Highlighting}
\end{Shaded}

{[}1{]}
``C:\textbackslash Users\textbackslash willi\textbackslash Desktop\textbackslash Pluralsight
Test\textbackslash model''

\begin{Shaded}
\begin{Highlighting}[]
\CommentTok{\#Take a look at confusion matrix. By default,}
\CommentTok{\#this will be based on training data}

\CommentTok{\#Note that h2o switches 0 and 1 from their conventional positions }
\CommentTok{\#(0/1 instead of 1/0)}
\FunctionTok{h2o.confusionMatrix}\NormalTok{(forest\_best)}
\end{Highlighting}
\end{Shaded}

Confusion Matrix (vertical: actual; across: predicted) for max f1 @
threshold = 0.386668797917577: 0 1 Error Rate 0 399 116 0.225243
=116/515 1 63 215 0.226619 =63/278 Totals 462 331 0.225725 =179/793

\begin{Shaded}
\begin{Highlighting}[]
\CommentTok{\#ROC Curve for GBM model based on cross validation }
\FunctionTok{plot}\NormalTok{(}\FunctionTok{h2o.performance}\NormalTok{(forest\_best, }\AttributeTok{xval =}\NormalTok{ T) ,}\AttributeTok{type=}\StringTok{\textquotesingle{}roc\textquotesingle{}}\NormalTok{)}
\end{Highlighting}
\end{Shaded}

\includegraphics{model_files/figure-latex/models-1.pdf}

\end{document}
